\documentclass[12pt]{article}

\usepackage[utf8]{inputenc}
%\usepackage[T1]{fontenc}

\usepackage{geometry}
\geometry{a4paper}
\usepackage{graphicx}
\usepackage{float}
\usepackage[italian]{babel}

%
%%%%%%%%%%%%%%%%%%%%%%%%%%%%%%%%%%%%%%%%%comando per la gestione semplificata di quotes
\newcommand{\quotes}[1]{``#1''}

%
%%%%%%%%%%%%%%%%%%%%%%%%%%%%%%%%%%%%%%%%%libreria per l'inserimento di link nella
%   bibliografia
\PassOptionsToPackage{hyphens}{url}\usepackage[hidelinks]{hyperref}

\linespread{1.2}
\setlength{\parindent}{0pt}

\begin{document}

%----------------------------------------------------------------------------------------
%	TITOLO
%----------------------------------------------------------------------------------------

\begin{titlepage}

\newcommand{\HRule}{\rule{\linewidth}{0.5mm}}

\center

\textsc{\Large Relazione di progetto di \quotes{Smart City e Tecnologie Mobili}}\\[0.5cm]

\HRule \\[0.4cm]
{
	\huge \bfseries
	Rilevatore di sonnolenza\\
	all'interno di un autoveicolo con\\
	Raspberry Pi\\[0.4cm]
}
\HRule \\[1.5cm]

\vfill

\begin{flushleft}
\emph{Numero del gruppo:}\\
62\\[1cm]
\emph{Componenti del gruppo:}\\
Giacomo Frisoni\\
Marcin Pabich\\[3cm]
\end{flushleft}



\end{titlepage}

%----------------------------------------------------------------------------------------
%	INDICE
%----------------------------------------------------------------------------------------

\tableofcontents

\newpage

%----------------------------------------------------------------------------------------
%	INTRODUZIONE
%----------------------------------------------------------------------------------------

\section{Introduzione}

Il deterioramento delle capacità alla guida causato da sonnolenza è noto per essere uno dei fattori che contribuiscono maggiormente alla formazione di incidenti automobilistici. Secondo un sondaggio del 2011 realizzato dalla National Sleep Foundation sulla popolazione americana, circa il 30\% degli incidenti su strada è dovuto all'affaticamento del conducente e tale dato tende a crescere di anno in anno \cite{SleepInAmerica}.\\
La sonnolenza costituisce nello specifico una fase di transizione nel ciclo sonno-veglia, determinante uno stato di torpore e una riduzione del livello di coscienza del guidatore. Essa può essere dovuta a diversi fattori: condizioni di guida avverse, traffico intenso, alti carichi di lavoro, scarso riposo, orari notturni, abuso di alcol e assunzione di medicinali sono solo alcuni esempi.\\
Misurare direttamente il livello di sonnolenza è complesso, ma esistono tuttavia numerosi metodi indiretti capaci di rilevare aspetti correlati e che oggi costituiscono la base per i vari sistemi disponibili sul mercato. Le misure di segnali fisiologici sono le più accurate ma comprendono tecniche quali ECG (elettrocardiogramma), EOG (elettrooculogramma) ed EEG (elettroencefalogramma) che faticano a essere adottate nella realtà a causa della connessione fisica che richiedono con l'autista.\\
La maggior parte delle soluzioni attualmente disponibili e impiegate dalle case automobilistiche riguardano misure basate sul veicolo e comprendono pertanto l'individuazione di scostamenti rispetto alla propria corsia, la pressione esercitata sui pedali e il movimento delle ruote sterzanti.\\
Negli ultimi anni, invece, si stanno sempre più diffondendo soluzioni basate su misure di tipo comportamentale che consentono in genere di rilevare la sonnolenza alla guida in modo precoce. Queste misurazioni si concentrono sul movimento degli occhi, sull'oscillazione della testa e sui cambiamenti facciali. Una persona assonnata, infatti, ha la tendenza a sbattere le palpebre frequentemente, a chiudere gli occhi, a sbadigliare ripetutamente e a inclinare la testa di una certa angolatura, ad esempio.\\
Il progetto descritto in questo documento si pone l'obiettivo di realizzare un sistema a basso costo, affidabile e non intrusivo per il riconoscimento della stanchezza del conducente in tempo reale, svolgendo un monitoraggio video sul suo volto e applicando tecniche di Computer Vision sui frame catturati. L'implementazione della soluzione si basa sull'uso di un \textit{Raspberry Pi 3 B+}, un single-board computer noto per le sue caratteristiche prestazionali rapportate al prezzo e alle dimensioni che lo contraddistinguono.\\
Il sistema acquisisce le immagini dell'utente per mezzo di una camera installata nell'automobile e posizionata frontalmente a esso. La sonnolenza del guidatore è successivamente calcolata in riferimento a una misura comportamentale incentrata sulla chiusura degli occhi (stimata quantitativamente con metodologia \textit{EAR}\footnote{EAR. Eye Aspect Ratio.}\cite{EAR}). A fronte di una chiusura degli occhi protrattasi oltre un tempo definito, il sistema riproduce un allarme per mezzo di un buzzer acustico al fine di reclamare l'attenzione dell'utente ed evidenziare il potenziale pericolo che sta incorrendo.\\
Le principali tecnologie adottate sono \textit{Python} a livello di linguaggio di programmazione, \textit{OpenCV} come libreria per la manipolazione di foto e video, \textit{dlib} per quanto concerne gli algoritmi di machine learning per l'individuazione e la localizzazione di facial landmark.\\
Il contributo tecnologico/scientifico apportato dal gruppo con l'elaborato in esame si riferisce allo studio e al confronto prestazionale delle varie soluzioni che possono adottate, specie in relazione ai limiti dell'hardware adottato. Una particolare attenzione è riposta nelle tecniche di riconoscimento facciale alla base, note per essere l'elemento centrale sia per la determinazione dell'efficacia del sistema nei vari scenari d'uso che per la sua reale efficienza.

\newpage


%----------------------------------------------------------------------------------------
%	STATO DELL'ARTE
%----------------------------------------------------------------------------------------

\section{Stato dell'arte}

Riassumere le soluzioni presenti in letteratura inerenti al problema in esame. Per ciascuna, discutere le principali diversità o affinità rispetto al progetto presentato. Nel caso non siano presenti soluzioni direttamente comparabili a quella presentata descrivere comunque le principali tecniche note per affrontare la tematica trattata.\\

Le soluzioni esposte devono essere corredate degli opportuni riferimenti bibliografici. Nel caso si tratti di soluzioni già operative sul mercato, devono essere indicate le fonti (online) dove poter accedere al servizio o approfondirne i contenuti.\\


Vincoli circa la lunghezza della sezione (escluse didascalie, tabelle, testo nelle immagini, schemi):

\vspace{1cm}
\begin{tabular}{l|rr}
 & Numero minimo di battute & Numero massimo di battute \\
 \hline
 1 componente & 2000 & 3000 \\
 2 componenti & 2500 & 4500 \\
 3 componenti & 3000 & 6000 \\
 \hline
\end{tabular}


\newpage


%----------------------------------------------------------------------------------------
%	ANALISI DEI REQUISITI
%----------------------------------------------------------------------------------------

\section{Analisi dei requisiti}

In questa sezione vengono trattati in modo dettagliato tutti i requisiti del progetto.

\begin{itemize}
	\item lo scopo generale del sistema è quello di riconoscere il grado di sonnolenza di un conducente di una macchina,
	\item per attivare/disattivare il dispositivo verrà messo a disposizione un bottone,
	\item per cominciare la rilevazione e l'eventuale segnalazione, il conducente deve preoccuparsi di far partire il rilevamento,
	\item i dati verranno catturati attraverso un dispositivo di cattura video a colori (una webcam),
	\item quest'ultimi verranno elaborati attraverso una piattaforma mobile a basso costo (Raspberry Pi 3 B+),
	\item non ci sono particolari vincoli sui linguaggi di programmazione utilizzati o sul hardware / software legacy da supportare, se non diversi dai limiti imposti dalla piattaforma stessa,
	\item per essere realizzabile e replicabile, il progetto deve mantenere un costo basso per tutti i componenti, utilizzati e garantire performance in real-time il più possibile accurate, con piccoli margini di errore,
	\item l'analisi dei dati consisterà nell'individuazione del volto del conducente tramite uno dei metodi noti in letteratura (HOG, HAAR, Linear SVM),
	\item gli algoritmi per l'individuazione del volto verranno testati e configurati, al fine di selezionare quello con migliori performance, dati i limiti della piattaforma mobile scelta,
	\item individuato il volto, si procederà alla corretta interpretazione della posizione degli occhi (facial lendmark) e il controllo dell'apertura degli occhi,
	\item la rilevazione degli occhi, e di conseguenza l'eventuale avviso, scatterà soltanto se il volto è correttamente riconosciuto,
	\item non vi sarà nessun avviso nel caso in cui il volto non verrà più correttamente riconosciuto, per eliminare falsi positivi dovuti, ad esempio, al conducente girato per fare retromarcia,
	\item soltanto se entrambi gli occhi risulteranno essere chiusi il conducente verrà allertato dal sistema,
	\item si prevedono due tipi di avvertimento: uno acustico (tramite buzzer) e uno luminoso (tramite LED),
	\item il sistema dev'essere robusto ad eventuali variazioni dell'ambiente, della luce e del posizionamento del volto,
	\item ove possibile, l'oggettistica o gli indumenti indossati non devono compromettere il corretto funzionamento del sistema,
	\item l'intera soluzione dev'essere facile da trasportare e poco ingombrande, per permettere un'installazione semplificata dentro il veicolo.
\end{itemize}

\iffalse
Il sistema progettato  dev'essere in grado di riconoscere il grado di sonnolenza di un conducente di una macchina. I dati provenienti da un dispositivo di cattura video a colori (una webcam) verranno elaborati attraverso una piattaforma mobile a basso costo (Raspberry Pi 3 B+). Non ci sono vincoli riguardanti librerie o linguaggi da utilizzare, come non ci sono vincoli sul hardware / software legacy da supportare. Comunque, per essere realizzabile e replicabile in grande scala, il progetto deve mantenere un costo basso per tutti i componenti utilizzati, ma deve garantire anche performance real-time il più possibile accurate, con piccoli margini di errore.\\

L'analisi dei dati consisterà nell'individuazione del volto del conducente tramite uno dei metodi noti in letteratura (HOG, HAAR, Linear SVM) opportunamente testati e configurati. L'algoritmo con risultati più soddisfacenti permetterà, successivamente, la corretta interpretazione della posizione degli occhi (facial lendmark) e il controllo dell'apertura degli occhi. Nel caso della rilevazione della chiusura di entrambi gli occhi, il conducente verrà allertato del pericolo tramite un segnale acustico e/o luminoso.\\

Il sistema dev'essere robusto alle eventuali variazioni dell'ambiente, quali luce o posizionamento del volto. Eventuale oggettistica indossata o il colore della pelle non deve causare problemi sulla corretta rilevazione. Quest'ultima inizierà e potrà essere interrotta tramite un bottone di accensione / spegnimento e partirà soltanto quando il volto del conducente sarà stato correttamente individuato. Il progetto, nell'insieme, dev'essere facile da trasportare e poco ingombrande, per permettere un'installazione semplificata dentro il veicolo.\\


In questa sezione esporre brevemente i requisiti a cui il sistema proposto deve rispondere, concentrando l'attenzione sugli aspetti più rilevanti e facendo eventualmente uso di opportuni diagrammi di alto livello.\\

Vincoli circa la lunghezza della sezione (escluse didascalie, tabelle, testo nelle immagini, schemi):

\vspace{1cm}
\begin{tabular}{l|rr}
 & Numero minimo di battute & Numero massimo di battute \\
 \hline
 1 componente & 4000 & 6000 \\
 2 componenti & 6000 & 8000 \\
 3 componenti & 8000 & 10000 \\
 \hline
\end{tabular}
\fi


\newpage


%----------------------------------------------------------------------------------------
%	PROGETTAZIONE
%----------------------------------------------------------------------------------------

\section{Progettazione}

Devono essere esposte le scelte progettuali operate nelle varie fasi di sviluppo dell'elaborato.\\

In questa sezione devono essere documentati gli schemi di progetto relativamente all'architettura complessiva del sistema e alle sue componenti di rilievo che possano meritare un'analisi di dettaglio. Per le componenti software si può ricorrere ad esempio a diagrammi delle classi, di sequenza, stato, attività. Per le componenti hardware è possibile includere opportuni schemi in grado di descrivere l'architettura fisica adottata.\\

Vincoli circa la lunghezza della sezione (escluse didascalie, tabelle, testo nelle immagini, schemi):

\vspace{1cm}
\begin{tabular}{l|rr}
 & Numero minimo di battute & Numero massimo di battute \\
 \hline
 1 componente & 9000 & 18000 \\
 2 componenti & 12000 & 21000 \\
 3 componenti & 15000 & 24000 \\
 \hline
\end{tabular}


\newpage


%----------------------------------------------------------------------------------------
%	IMPLEMENTAZIONE
%----------------------------------------------------------------------------------------

\section{Implementazione}\label{sec:implementazione}

Esporre i principali problemi affrontati durante l'effettiva realizzazione delle componenti hardware/software e illustrare le soluzioni implementative adottate. Se l'elaborato ha previsto l'utilizzo di tecnologie già disponibili sul mercato, discuterne brevemente le caratteristiche e motivarne l'adozione rispetto ad altre soluzioni assimilabili.\\

\textbf{NOTA: in questa sezione devono essere riportate esclusivamente le porzioni di codice ritenute particolarmente significative. Il codice sorgente nella sua interezza, opportunamente commentato, deve essere consegnato separatamente dalla relazione in un archivio compresso.}\\


Vincoli circa la lunghezza della sezione (escluse didascalie, tabelle, testo nelle immagini, schemi):

\vspace{1cm}
\begin{tabular}{l|rr}
 & Numero minimo di battute & Numero massimo di battute \\
 \hline
 1 componente & 5000 & 11000 \\
 2 componenti & 8000 & 16000 \\
 3 componenti & 10000 & 21000 \\
 \hline
\end{tabular}


\newpage


%----------------------------------------------------------------------------------------
%	TESTING E PERFORMANCE
%----------------------------------------------------------------------------------------

\section{Testing e performance}

Esporre lo stato di funzionamento effettivo del sistema progettato ad elaborato concluso. Per ciascuna delle funzionalità salienti devono essere tabellate e discusse le performance riscontrate mediante opportuni test eseguiti in fase di validazione del progetto.\\

I tempi di esecuzione/comunicazione devono essere accompagnati dalle caratteristiche dell'hardware sul quale è eseguito il software.\\

Qualora l'elaborato includa algoritmi innovativi, indicarne la complessità computazionale (avendo cura di esporre lo pseudo codice nella sezione \ref{sec:implementazione}).\\


Vincoli circa la lunghezza della sezione (escluse didascalie, tabelle, testo nelle immagini, schemi):

\vspace{1cm}
\begin{tabular}{l|rr}
 & Numero minimo di battute & Numero massimo di battute \\
 \hline
 1 componente & 2000 & 3000 \\
 2 componenti & 2500 & 4500 \\
 3 componenti & 3000 & 6000 \\
 \hline
\end{tabular}


\newpage


%----------------------------------------------------------------------------------------
%	ANALISI DI DEPLOYMENT SU LARGA SCALA
%----------------------------------------------------------------------------------------

\section{Analisi di deployment su larga scala}

In questa sezione va discussa, eventualmente con l'ausilio di opportuni diagrammi (componenti, deployment), l'evoluzione del progetto presentato immaginando che venga adottato su larga scala. I dettagli qui esposti devono quindi astrarre dalle specifiche dell'elaborato qualora l'implementazione sia stata focalizzata su uno scenario isolato.\\

A titolo d’esempio, qualora applicabile, devono essere evidenziate le criticità che si potrebbero incontrare e devono essere proposte soluzioni tipiche in contesti di \textit{cloud architecture} per garantire un'adeguata \textit{resilienza}, in termini di \textit{availability} e \textit{scalability} del sistema.\\


Vincoli circa la lunghezza della sezione (escluse didascalie, tabelle, testo nelle immagini, schemi):

\vspace{1cm}
\begin{tabular}{l|rr}
 & Numero minimo di battute & Numero massimo di battute \\
 \hline
 1 componente & 3000 & 6000 \\
 2 componenti & 4500 & 9000 \\
 3 componenti & 6000 & 12000 \\
 \hline
\end{tabular}


\newpage


%----------------------------------------------------------------------------------------
%	PIANO DI LAVORO
%----------------------------------------------------------------------------------------

\section{Piano di lavoro}

In questa sezione devono essere chiariti i compiti svolti da ciascun candidato nel caso in cui il gruppo abbia più di un componente.\\

Deve essere inoltre esposto il piano di lavoro adottato. A tal fine, per ogni attività svolta durante la preparazione dell'elaborato (ad esempio: studio di una tecnologia, progettazione di un componente, implementazione di un algoritmo ecc…) deve essere chiarita la collocazione temporale e devono essere indicate le risorse impiegate per svolgerla (giorni/uomo). I candidati possono ricorrere a opportuni diagrammi come quello di Gantt.\\


Vincoli circa la lunghezza della sezione (escluse didascalie, tabelle, testo nelle immagini, schemi):

\vspace{1cm}
\begin{tabular}{l|rr}
 & Numero minimo di battute & Numero massimo di battute \\
 \hline
 1 componente & 1000 & 2000 \\
 2 componenti & 1500 & 3000 \\
 3 componenti & 2000 & 4000 \\
 \hline
\end{tabular}

\newpage


%----------------------------------------------------------------------------------------
%	CONCLUSIONI
%----------------------------------------------------------------------------------------

\section{Conclusioni}

Esporre brevemente le considerazioni conclusive sul progetto presentato, indicando anche i possibili sviluppi futuri.\\

Vincoli circa la lunghezza della sezione (escluse didascalie, tabelle, testo nelle immagini, schemi):

\vspace{1cm}
\begin{tabular}{l|rr}
 & Numero minimo di battute & Numero massimo di battute \\
 \hline
 1 componente & 500 & 1000 \\
 2 componenti & 1000 & 2000 \\
 3 componenti & 1500 & 3000 \\
 \hline
\end{tabular}

\newpage


%----------------------------------------------------------------------------------------
%	APPENDICE
%----------------------------------------------------------------------------------------

\appendix
\addcontentsline{toc}{section}{Appendice}
\section*{Appendice}
Laddove necessario è possibile avvalersi di appendici alla relazione per includere materiale di approfondimento.\\

A titolo esemplificativo possono essere incluse le schede tecniche dei componenti adottati, la normativa di riferimento che regola un particolare dominio applicativo, ecc.


\newpage


%----------------------------------------------------------------------------------------
%	RIFERIMENTI BIBLIOGRAFICI
%----------------------------------------------------------------------------------------

\bibliography{relazione}
\bibliographystyle{unsrt}

%----------------------------------------------------------------------------------------

\end{document}