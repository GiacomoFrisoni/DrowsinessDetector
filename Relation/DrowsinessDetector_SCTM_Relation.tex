\documentclass[12pt]{article}

\usepackage[utf8]{inputenc}
%\usepackage[T1]{fontenc}

\usepackage{geometry}
\geometry{a4paper}
\usepackage{graphicx}
\usepackage{float}
\usepackage[italian]{babel}

%
%%%%%%%%%%%%%%%%%%%%%%%%%%%%%%%%%%%%%%%%%comando per la gestione semplificata di quotes
\newcommand{\quotes}[1]{``#1''}

%
%%%%%%%%%%%%%%%%%%%%%%%%%%%%%%%%%%%%%%%%%libreria per l'inserimento di link nella
%   bibliografia
\PassOptionsToPackage{hyphens}{url}\usepackage[hidelinks]{hyperref}

\linespread{1.2}
\setlength{\parindent}{0pt}

\begin{document}

%----------------------------------------------------------------------------------------
%	TITOLO
%----------------------------------------------------------------------------------------

\begin{titlepage}

\newcommand{\HRule}{\rule{\linewidth}{0.5mm}}

\center

\textsc{\Large Relazione di progetto di \quotes{Smart City e Tecnologie Mobili}}\\[0.5cm]

\HRule \\[0.4cm]
{
	\huge \bfseries
	Rilevatore di sonnolenza\\
	all'interno di un autoveicolo con\\
	Raspberry Pi\\[0.4cm]
}
\HRule \\[1.5cm]

\vfill

\begin{flushleft}
\emph{Numero del gruppo:}\\
62\\[1cm]
\emph{Componenti del gruppo:}\\
Giacomo Frisoni\\
Marcin Pabich\\[3cm]
\end{flushleft}



\end{titlepage}

%----------------------------------------------------------------------------------------
%	INDICE
%----------------------------------------------------------------------------------------

\tableofcontents

\newpage

%----------------------------------------------------------------------------------------
%	INTRODUZIONE
%----------------------------------------------------------------------------------------

\section{Introduzione}

Il deterioramento delle capacità alla guida causato da sonnolenza è noto per essere uno dei fattori che contribuiscono maggiormente alla formazione di incidenti automobilistici. Secondo un sondaggio del 2011 realizzato dalla National Sleep Foundation sulla popolazione americana, circa il 30\% degli incidenti su strada è dovuto all'affaticamento del conducente e tale dato tende a crescere di anno in anno \cite{SleepInAmerica}.\\
La sonnolenza costituisce nello specifico una fase di transizione nel ciclo sonno-veglia, determinante uno stato di torpore e una riduzione del livello di coscienza del guidatore. Essa può essere dovuta a diversi fattori: condizioni di guida avverse, traffico intenso, alti carichi di lavoro, scarso riposo, orari notturni, abuso di alcol e assunzione di medicinali sono solo alcuni esempi.\\
Misurare direttamente il livello di sonnolenza è complesso, ma esistono tuttavia numerosi metodi indiretti capaci di rilevare aspetti correlati e che oggi costituiscono la base per i vari sistemi disponibili sul mercato. Le misure di segnali fisiologici sono le più accurate ma comprendono tecniche quali ECG (elettrocardiogramma), EOG (elettrooculogramma) ed EEG (elettroencefalogramma) che faticano a essere adottate nella realtà a causa della connessione fisica che richiedono con l'autista.\\
La maggior parte delle soluzioni attualmente disponibili e impiegate dalle case automobilistiche riguardano misure basate sul veicolo e comprendono pertanto l'individuazione di scostamenti rispetto alla propria corsia, la pressione esercitata sui pedali e il movimento delle ruote sterzanti.\\
Negli ultimi anni, invece, si stanno sempre più diffondendo soluzioni basate su misure di tipo comportamentale che consentono in genere di rilevare la sonnolenza alla guida in modo precoce. Queste misurazioni si concentrono sul movimento degli occhi, sull'oscillazione della testa e sui cambiamenti facciali. Una persona assonnata, infatti, ha la tendenza a sbattere le palpebre frequentemente, a chiudere gli occhi, a sbadigliare ripetutamente e a inclinare la testa di una certa angolatura, ad esempio.\\
Il progetto descritto in questo documento si pone l'obiettivo di realizzare un sistema a basso costo, affidabile e non intrusivo per il riconoscimento della stanchezza del conducente in tempo reale, svolgendo un monitoraggio video sul suo volto e applicando tecniche di Computer Vision sui frame catturati. L'implementazione della soluzione si basa sull'uso di un \textit{Raspberry Pi 3 B+}, un single-board computer noto per le sue caratteristiche prestazionali rapportate al prezzo e alle dimensioni che lo contraddistinguono.\\
Il sistema acquisisce le immagini dell'utente per mezzo di una camera installata nell'automobile e posizionata frontalmente a esso. La sonnolenza del guidatore è successivamente calcolata in riferimento a una misura comportamentale incentrata sulla chiusura degli occhi (stimata quantitativamente con metodologia \textit{EAR}\footnote{EAR. Eye Aspect Ratio.}\cite{EAR}). A fronte di una chiusura degli occhi protrattasi oltre un tempo definito, il sistema riproduce un allarme per mezzo di un buzzer acustico al fine di reclamare l'attenzione dell'utente ed evidenziare il potenziale pericolo che sta incorrendo.\\
Le principali tecnologie adottate sono \textit{Python} a livello di linguaggio di programmazione, \textit{OpenCV} come libreria per la manipolazione di foto e video, \textit{dlib} per quanto concerne gli algoritmi di machine learning per l'individuazione e la localizzazione di facial landmark.\\
Il contributo tecnologico/scientifico apportato dal gruppo con l'elaborato in esame si riferisce allo studio e al confronto prestazionale delle varie soluzioni che possono adottate, specie in relazione ai limiti dell'hardware adottato. Una particolare attenzione è riposta nelle tecniche di riconoscimento facciale alla base, note per essere l'elemento centrale sia per la determinazione dell'efficacia del sistema nei vari scenari d'uso che per la sua reale efficienza.

\newpage


%----------------------------------------------------------------------------------------
%	STATO DELL'ARTE
%----------------------------------------------------------------------------------------

\section{Stato dell'arte}

Riassumere le soluzioni presenti in letteratura inerenti al problema in esame. Per ciascuna, discutere le principali diversità o affinità rispetto al progetto presentato. Nel caso non siano presenti soluzioni direttamente comparabili a quella presentata descrivere comunque le principali tecniche note per affrontare la tematica trattata.\\

Le soluzioni esposte devono essere corredate degli opportuni riferimenti bibliografici. Nel caso si tratti di soluzioni già operative sul mercato, devono essere indicate le fonti (online) dove poter accedere al servizio o approfondirne i contenuti.\\


Vincoli circa la lunghezza della sezione (escluse didascalie, tabelle, testo nelle immagini, schemi):

\vspace{1cm}
\begin{tabular}{l|rr}
 & Numero minimo di battute & Numero massimo di battute \\
 \hline
 1 componente & 2000 & 3000 \\
 2 componenti & 2500 & 4500 \\
 3 componenti & 3000 & 6000 \\
 \hline
\end{tabular}


\newpage


%----------------------------------------------------------------------------------------
%	ANALISI DEI REQUISITI
%----------------------------------------------------------------------------------------

\section{Analisi dei requisiti}

In questa sezione sono trattati in modo dettagliato tutti i requisiti del progetto, emersi durante la fase di analisi.

\subsection{Business Requirements}
\begin{enumerate}
	\item Realizzare un progetto di qualità sia nella sua parte software che hardware, valutandone accuratamente le prestazioni.
	\item Organizzare il lavoro in team, definendo un apposito piano e sperimentando la suddivisione di task mediante diagramma di Gantt.
	\item Mettere alla prova le conoscenze acquisite durante il corso relative ai sistemi embedded, al Raspberry Pi, alla videosorveglianza e alle tecnologie di sensing.
\end{enumerate}

\subsection{User Requirements}
\begin{enumerate}
	\item Possibilità di posizionare il dispositivo all'interno del veicolo, frontalmente al conducente.
	\item Capacità del sistema di riconoscere il livello di sonnolenza del guidatore, considerando una misura comportamentale basata sulla chiusura degli occhi.
	\begin{enumerate}
		\item Il sistema deve considerare il guidatore in uno stato di sonnolenza e pertanto in una situazione di pericolo se la chiusura dei suoi occhi si protrae per un tempo eccessivo.
		\item Nella misurazione della sonnolenza, il sistema deve tener conto della reale apertura di ambo gli occhi.
	\end{enumerate}
	\item Avvio automatico del sistema all'atto dell'alimentazione.
	\item Attivazione o disattivazione del rilevamento su richiesta del conducente.
	\begin{enumerate}
		\item Il sistema deve procedere al monitoraggio del volto del conducente, al rilevamento del suo grado di sonnolenza e alla segnalazione di scenari di pericolo solo se attivo.
	\end{enumerate}
	\item Gestione degli scenari d'uso legati al riconoscimento del volto.
	\begin{enumerate}
		\item Volto rilevato. Il sistema deve procedere con la misura dell'indice di sonnolenza.
		\item Volto non rilevato. Il sistema non deve produrre allarmi, evitando segnalazioni potenzialmente non connesse a situazioni di reale pericolosità.
	\end{enumerate}
	\item Segnalazione acustica e visiva degli scenari di pericolo rilevati.
\end{enumerate}

\subsection{Functional Requirements}
\begin{enumerate}
	\item Il sistema deve monitorare il viso dell'utente attraverso una camera.
	\item La misurazione della sonnolenza del conducente deve avvenire grazie a tecniche di Computer Vision.
	\begin{enumerate}
		\item Il sistema deve applicare tecniche di face recognition per l'individuazione del volto.
		\item Il sistema deve applicare tecniche di facial landmarking per l'individuazione delle sole coordinate di interesse per quanto concerne gli occhi.
		\item Il sistema deve rilevare la chiusura degli occhi nei vari frame grazie al calcolo dell'EAR per ognuno di essi.
		\item Si considera la presenza di uno scenario di pericolo nel momento in cui la misura dell'EAR indica la chiusura degli occhi per un definito numero di frame consecutivi, ritenuto di adatta sensibilità per gli obiettivi del sistema.
	\end{enumerate}
	\item Il sistema deve avviare il software di rilevamento nella fase di reboot.
	\item Il sistema deve consentire l'avvio o l'arresto della rilevazione mediante la pressione di un pulsante.
	\begin{enumerate}
		\item L'acquisizione dei frame dalla camera e la loro successiva elaborazione devono avvenire solo se il sistema è attivo.
	\end{enumerate}
	\item Il sistema deve adottare un comportamento idoneo sia a fronte del riconoscimento del volto che del fallimento di tale fase.
	\begin{enumerate}
		\item Volto rilevato. Il sistema deve procedere con l'estrazione dei facial landmark e il calcolo dell'EAR per entrambi gli occhi.
		\item Volto non rilevato. Il sistema deve procedere con l'analisi dei successivi frame, senza segnalare situazioni di pericolo.
	\end{enumerate}
	\item Il sistema deve segnalare al conducente la presenza di uno scenario di pericolo causato dal rilevamento di sonnolenza, adottando due tipologie distinte di avvertimento.
	\begin{enumerate}
		\item Avvertimento acustico, per mezzo di un buzzer sonoro.
		\item Avvertimento visivo, per mezzo di LED.
	\end{enumerate}
\end{enumerate}

\subsection{Non-functional Requirements}
\begin{enumerate}
	\item \textbf{Dimensioni.} Il sistema deve avere dimensioni conformi per una sua installazione a bordo di un autoveicolo, in modo da non ostacolare in alcun modo la guida. L'intera soluzione deve pertanto essere facile da trasportare e poco ingombrante.
	\item \textbf{Real-time.} Il sistema deve essere in grado di elaborare i frame catturati e di fornire segnalazioni di pericolo in tempo reale, specie in considerazione della velocità con cui il veicolo potrebbe essere in movimento. Ciò ha impatto sia sulla scelta dell'hardware - che deve risultare di sufficiente capacità computazionale - sia sulla progettazione del software - che non deve rivelarsi eccessivamente complesso nel rispetto delle limitate risorse che si hanno a disposizione sulla piattaforma scelta.
	\item \textbf{Affidabilità.} Il rilevamento della sonnolenza del conducente deve avvenire nel modo più accurato possibile, riducendo al minimo il numero di falsi positivi e consentendo solo ridotti margini di errore. Il sistema deve essere robusto a eventuali variazioni dell'ambiente, della luminosità e dell'orientamento del volto. Ove possibile, inoltre, l'oggettistica, gli indumenti indossati e la colorazione della pelle del guidatore non devono compromettere il funzionamento del sistema stesso.
\end{enumerate}

\subsection{Implementation Requirements}
\begin{enumerate}
	\item \textbf{Raspberry Pi 3 B+}. La componente hardware del sistema deve essere basata su Raspberry Pi 3 B+, un single-board computer economico, di ridotte dimensioni e di buone capacità computazionali. La ragione alla base di tale scelta è da ricercarsi nella volontà di approfondire le conoscenze apprese durante il corso di Smart City e Tecnologie Mobili, acquisendo anche esperienza nello sviluppo di progetti basati su questo tipo di piattaforma.
	\item \textbf{Budget}. La realizzazione del sistema deve avvenire a basso costo da un punto di vista economico in tutte le sue componenti, nell'ottica anche di un ipotetico deployment su larga scala.
\end{enumerate}

\iffalse
In questa sezione esporre brevemente i requisiti a cui il sistema proposto deve rispondere, concentrando l'attenzione sugli aspetti più rilevanti e facendo eventualmente uso di opportuni diagrammi di alto livello.\\

Vincoli circa la lunghezza della sezione (escluse didascalie, tabelle, testo nelle immagini, schemi):

\vspace{1cm}
\begin{tabular}{l|rr}
 & Numero minimo di battute & Numero massimo di battute \\
 \hline
 1 componente & 4000 & 6000 \\
 2 componenti & 6000 & 8000 \\
 3 componenti & 8000 & 10000 \\
 \hline
\end{tabular}
\fi


\newpage


%----------------------------------------------------------------------------------------
%	PROGETTAZIONE
%----------------------------------------------------------------------------------------

\section{Progettazione}

Devono essere esposte le scelte progettuali operate nelle varie fasi di sviluppo dell'elaborato.\\

In questa sezione devono essere documentati gli schemi di progetto relativamente all'architettura complessiva del sistema e alle sue componenti di rilievo che possano meritare un'analisi di dettaglio. Per le componenti software si può ricorrere ad esempio a diagrammi delle classi, di sequenza, stato, attività. Per le componenti hardware è possibile includere opportuni schemi in grado di descrivere l'architettura fisica adottata.\\

Vincoli circa la lunghezza della sezione (escluse didascalie, tabelle, testo nelle immagini, schemi):

\vspace{1cm}
\begin{tabular}{l|rr}
 & Numero minimo di battute & Numero massimo di battute \\
 \hline
 1 componente & 9000 & 18000 \\
 2 componenti & 12000 & 21000 \\
 3 componenti & 15000 & 24000 \\
 \hline
\end{tabular}


\newpage


%----------------------------------------------------------------------------------------
%	IMPLEMENTAZIONE
%----------------------------------------------------------------------------------------

\section{Implementazione}\label{sec:implementazione}

Esporre i principali problemi affrontati durante l'effettiva realizzazione delle componenti hardware/software e illustrare le soluzioni implementative adottate. Se l'elaborato ha previsto l'utilizzo di tecnologie già disponibili sul mercato, discuterne brevemente le caratteristiche e motivarne l'adozione rispetto ad altre soluzioni assimilabili.\\

\textbf{NOTA: in questa sezione devono essere riportate esclusivamente le porzioni di codice ritenute particolarmente significative. Il codice sorgente nella sua interezza, opportunamente commentato, deve essere consegnato separatamente dalla relazione in un archivio compresso.}\\


Vincoli circa la lunghezza della sezione (escluse didascalie, tabelle, testo nelle immagini, schemi):

\vspace{1cm}
\begin{tabular}{l|rr}
 & Numero minimo di battute & Numero massimo di battute \\
 \hline
 1 componente & 5000 & 11000 \\
 2 componenti & 8000 & 16000 \\
 3 componenti & 10000 & 21000 \\
 \hline
\end{tabular}


\newpage


%----------------------------------------------------------------------------------------
%	TESTING E PERFORMANCE
%----------------------------------------------------------------------------------------

\section{Testing e performance}

Esporre lo stato di funzionamento effettivo del sistema progettato ad elaborato concluso. Per ciascuna delle funzionalità salienti devono essere tabellate e discusse le performance riscontrate mediante opportuni test eseguiti in fase di validazione del progetto.\\

I tempi di esecuzione/comunicazione devono essere accompagnati dalle caratteristiche dell'hardware sul quale è eseguito il software.\\

Qualora l'elaborato includa algoritmi innovativi, indicarne la complessità computazionale (avendo cura di esporre lo pseudo codice nella sezione \ref{sec:implementazione}).\\


Vincoli circa la lunghezza della sezione (escluse didascalie, tabelle, testo nelle immagini, schemi):

\vspace{1cm}
\begin{tabular}{l|rr}
 & Numero minimo di battute & Numero massimo di battute \\
 \hline
 1 componente & 2000 & 3000 \\
 2 componenti & 2500 & 4500 \\
 3 componenti & 3000 & 6000 \\
 \hline
\end{tabular}


\newpage


%----------------------------------------------------------------------------------------
%	ANALISI DI DEPLOYMENT SU LARGA SCALA
%----------------------------------------------------------------------------------------

\section{Analisi di deployment su larga scala}

In questa sezione va discussa, eventualmente con l'ausilio di opportuni diagrammi (componenti, deployment), l'evoluzione del progetto presentato immaginando che venga adottato su larga scala. I dettagli qui esposti devono quindi astrarre dalle specifiche dell'elaborato qualora l'implementazione sia stata focalizzata su uno scenario isolato.\\

A titolo d’esempio, qualora applicabile, devono essere evidenziate le criticità che si potrebbero incontrare e devono essere proposte soluzioni tipiche in contesti di \textit{cloud architecture} per garantire un'adeguata \textit{resilienza}, in termini di \textit{availability} e \textit{scalability} del sistema.\\


Vincoli circa la lunghezza della sezione (escluse didascalie, tabelle, testo nelle immagini, schemi):

\vspace{1cm}
\begin{tabular}{l|rr}
 & Numero minimo di battute & Numero massimo di battute \\
 \hline
 1 componente & 3000 & 6000 \\
 2 componenti & 4500 & 9000 \\
 3 componenti & 6000 & 12000 \\
 \hline
\end{tabular}


\newpage


%----------------------------------------------------------------------------------------
%	PIANO DI LAVORO
%----------------------------------------------------------------------------------------

\section{Piano di lavoro}

In questa sezione devono essere chiariti i compiti svolti da ciascun candidato nel caso in cui il gruppo abbia più di un componente.\\

Deve essere inoltre esposto il piano di lavoro adottato. A tal fine, per ogni attività svolta durante la preparazione dell'elaborato (ad esempio: studio di una tecnologia, progettazione di un componente, implementazione di un algoritmo ecc…) deve essere chiarita la collocazione temporale e devono essere indicate le risorse impiegate per svolgerla (giorni/uomo). I candidati possono ricorrere a opportuni diagrammi come quello di Gantt.\\


Vincoli circa la lunghezza della sezione (escluse didascalie, tabelle, testo nelle immagini, schemi):

\vspace{1cm}
\begin{tabular}{l|rr}
 & Numero minimo di battute & Numero massimo di battute \\
 \hline
 1 componente & 1000 & 2000 \\
 2 componenti & 1500 & 3000 \\
 3 componenti & 2000 & 4000 \\
 \hline
\end{tabular}

\newpage


%----------------------------------------------------------------------------------------
%	CONCLUSIONI
%----------------------------------------------------------------------------------------

\section{Conclusioni}

Esporre brevemente le considerazioni conclusive sul progetto presentato, indicando anche i possibili sviluppi futuri.\\

Vincoli circa la lunghezza della sezione (escluse didascalie, tabelle, testo nelle immagini, schemi):

\vspace{1cm}
\begin{tabular}{l|rr}
 & Numero minimo di battute & Numero massimo di battute \\
 \hline
 1 componente & 500 & 1000 \\
 2 componenti & 1000 & 2000 \\
 3 componenti & 1500 & 3000 \\
 \hline
\end{tabular}

\newpage


%----------------------------------------------------------------------------------------
%	APPENDICE
%----------------------------------------------------------------------------------------

\appendix
\addcontentsline{toc}{section}{Appendice}
\section*{Appendice}
Laddove necessario è possibile avvalersi di appendici alla relazione per includere materiale di approfondimento.\\

A titolo esemplificativo possono essere incluse le schede tecniche dei componenti adottati, la normativa di riferimento che regola un particolare dominio applicativo, ecc.


\newpage


%----------------------------------------------------------------------------------------
%	RIFERIMENTI BIBLIOGRAFICI
%----------------------------------------------------------------------------------------

\bibliography{relazione}
\bibliographystyle{unsrt}

%----------------------------------------------------------------------------------------

\end{document}